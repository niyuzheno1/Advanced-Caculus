\documentclass{article}
\usepackage[utf8]{inputenc}
\usepackage{amsmath}

\title{More Calculus Haha}
\author{waluigi120 }
\date{April 2018}

\begin{document}

\maketitle

\section{Double Exponential}

\subsection{Definition}
$$y = \frac{1}{2\sigma} e^{-\frac{|x-\mu|}{\sigma}}$$

\subsection{Moment Generating Function}
\begin{align*}\label{eq:pareto mle2}
 Mgf(t) =\int_{-\infty}^{+\infty} e^{tx} \frac{1}{2\sigma} e^{-\frac{|x-\mu|}{\sigma}} dx\\ =\int_{-\infty}^{+\infty} \frac{1}{2\sigma} e^{tx-\frac{|x-\mu|}{\sigma}}dx \\ = \int_{-\infty}^{\mu} \frac{1}{2\sigma} e^{tx-\frac{\mu - x}{\sigma}} dx \\+ \int_{\mu}^{\infty} \frac{1}{2\sigma} e^{tx-\frac{x-\mu}{\sigma}} dx \\ = \int_{-\infty}^{\mu} \frac{1}{2\sigma} e^{tx-\frac{\mu - x}{\sigma}}dx \\+ \int_{\mu}^{\infty} \frac{1}{2\sigma} e^{tx-\frac{x-\mu}{\sigma}}dx\\ = \int_{-\infty}^{\mu} \frac{1}{2\sigma} e^{\frac{\sigma tx -\mu + x}{\sigma}}dx \\+ \int_{\mu}^{\infty} \frac{1}{2\sigma} e^{\frac{\sigma tx-x+\mu}{\sigma}}dx \\ = \frac{1}{2 \sigma}( \int_{-\infty}^{\mu}  e^{\frac{\sigma tx -\mu + x}{\sigma}}dx +  \int_{\mu}^{\infty}  e^{\frac{\sigma tx-x+\mu}{\sigma}}dx )\\ = \frac{1}{ 2\sigma} ( e^{-\frac{\mu}{\sigma}} \int_{-\infty}^{\mu} e^{\frac{(\sigma t + 1) x}{\sigma}} dx +  e^{\frac{\mu}{\sigma}} \int_{\mu}^{\infty}  e^{\frac{\sigma tx-x}{\sigma}}dx )) \\ =  \frac{1}{2 \sigma} (e^{-\frac{\mu}{\sigma}} \frac{\sigma}{\sigma t + 1} e^{\frac{(\sigma t + 1) x}{\sigma}} |_{x = -\infty}^{\mu}+e^{\frac{\mu}{\sigma}} \frac{\sigma}{\sigma t - 1} e^{\frac{(\sigma t - 1) x}{\sigma}} |_{x = \mu}^{\infty} ) \\ = \frac{1}{2\sigma} (e^{-\frac{\mu}{\sigma}} \frac{\sigma}{\sigma t + 1} e^{\frac{(\sigma t + 1) \mu}{\sigma}} - e^{\frac{\mu}{\sigma}} \frac{\sigma}{\sigma t - 1} e^{\frac{(\sigma t - 1) \mu}{\sigma}} )\end{align*}
\begin{align*}
 = \frac{1}{2\sigma} ( \frac{\sigma}{\sigma t + 1} e^{\frac{(\sigma t ) \mu}{\sigma}} -  \frac{\sigma}{\sigma t - 1} e^{\frac{(\sigma t) \mu}{\sigma}} ) =   \frac{1}{1-(\sigma t)^{2} } e^{ t  \mu}
\end{align*}

\subsection{Means and Variance}

Mean is $\mu$, which is easy to show.

Variance ,easy to be proven by substitution and take derivative, is $2\sigma^2$

\section{Fisher Distribution}

\subsection{Moments}
nth-Moment is defined by the following integral:

$$\int_{0}^{\infty}x^n\frac{\Gamma(\frac{\nu_1+\nu_2}{2})}{\Gamma(\frac{\nu_1}{2})\Gamma(\frac{\nu_2}{2})} (\frac{\nu_1}{\nu_2})^{\frac{\nu_1}{2}} \frac{x^{\frac{\nu_1-2}{2}}}{(1+(\frac{\nu_1}{\nu_2})x)^{\frac{\nu_1+\nu_2}{2}}}dx$$

$$\frac{\Gamma(\frac{\nu_1+\nu_2}{2})}{\Gamma(\frac{\nu_1}{2})\Gamma(\frac{\nu_2}{2})} (\frac{\nu_1}{\nu_2})^{\frac{\nu_1}{2}}\int_{0}^{\infty}x^{2n/2} \frac{x^{\frac{\nu_1-2}{2}}}{(1+(\frac{\nu_1}{\nu_2})x)^{\frac{\nu_1+\nu_2}{2}}}dx$$

Focusing on the integral,

$$\int_{0}^{\infty} \frac{x^{\frac{2n+\mu_1-2}{2}}}{(1+(\frac{\nu_1}{\nu_2})x)^{\frac{\nu_1+\nu_2}{2}}}dx$$


$$\int_{0}^{\infty} \frac{x^{\frac{2n+\nu_1-2}{2}}}{(1+(\frac{\nu_1}{\nu_2})x)^{\frac{\nu_1+\nu_2}{2}}}dx$$

Let $u = 1+(\frac{\nu_1}{\nu_2})x$ then $(u-1)(\frac{\nu_2}{\nu_1})= x$

and $du(\frac{\nu_2}{\nu_1})= dx$, if $x =0,u=1$, if $x=\infty, u = \infty$

$$\int_{1}^{\infty} \frac{((u-1)(\frac{\nu_2}{\nu_1}))^{\frac{2n+\nu_1-2}{2}}}{u^{\frac{\nu_1+\nu_2}{2}}}(\frac{\nu_2}{\nu_1})du$$

$$(\frac{\nu_2}{\nu_1})^{\frac{2n+\nu_1}{2}} \int_{1}^{\infty} \frac{(u-1)^{\frac{2n+\nu_1-2}{2}}}{u^{\frac{\nu_1+\nu_2}{2}}}du$$


$$(\frac{\nu_2}{\nu_1})^{\frac{2n+\nu_1}{2}} \int_{1}^{\infty} \frac{(u-1)^{\frac{2n+\nu_1-2}{2}}}{u^{\frac{\nu_1+\nu_2}{2}}}du$$
 
Focusing on the integral, Let $v =u^{-1}$, $dv = -u^{-2} du = -v^2du$

$du = -v^{-2} dv$

$$\int_{0}^{1} \frac{(v^{-1}-1)^{\frac{2n+\nu_1-2}{2}}}{{v^{-1}}^{\frac{\nu_1+\nu_2}{2}}}v^{-2} dv$$

$$\int_{0}^{1} (\frac{1-v}{v})^{\frac{2n+\nu_1-2}{2}}v^{-2} {{{v}^{\frac{\nu_1+\nu_2}{2}}}} dv$$

$$\int_{0}^{1} ({1-v})^{\frac{2n+\nu_1-2}{2}}v^{\frac{-2n-\nu_1+2}{2}}v^{-2} {{{v}^{\frac{\nu_1+\nu_2}{2}}}} dv$$

$$\int_{0}^{1} ({1-v})^{\frac{2n+\nu_1-2}{2}}v^{\frac{\nu_2-2n-2}{2}}{} dv$$

Which as it turns out to be, 
$${\bf B}(\frac{\nu_2-2n}{2},\frac{2n+\nu_1}{2})$$

Plug in back,

$$\frac{\Gamma(\frac{\nu_1+\nu_2}{2})}{\Gamma(\frac{\nu_1}{2})\Gamma(\frac{\nu_2}{2})} (\frac{\nu_1}{\nu_2})^{\frac{\nu_1}{2}}(\frac{\nu_2}{\nu_1})^{\frac{2n+\nu_1}{2}} {\bf B}(\frac{\nu_2-2n}{2},\frac{2n+\nu_1}{2})$$

Which will be,

$$\frac{\Gamma(\frac{\nu_2-2n}{2})\Gamma(\frac{2n+\nu_1}{2})}{\Gamma(\frac{\nu_1}{2})\Gamma(\frac{\nu_2}{2})} (\frac{\nu_2}{\nu_1})^{n} $$

\section{T-distribution}

\subsection{Moments}
The moments by T-distribution is defined by,

$$\int_{-\infty}^{\infty}\frac{\Gamma(\frac{\nu+1}{2})}{\Gamma(\frac{\nu}{2})} \frac{1}{\sqrt{\nu \pi}} \frac{x^n}{(1+\frac{x^2}{\nu})^{\frac{\nu+1}{2}}}dx$$

$$\frac{\Gamma(\frac{\nu+1}{2})}{\Gamma(\frac{\nu}{2})} \frac{1}{\sqrt{\nu \pi}} \int_{-\infty}^{\infty} \frac{x^n}{(1+\frac{x^2}{\nu})^{\frac{\nu+1}{2}}}dx$$

Focusing on the integral, 

$$\int_{-\infty}^{\infty} \frac{x^n}{(1+\frac{x^2}{\nu})^{\frac{\nu+1}{2}}}dx$$

Can be rewritten as,

$$2 \int_{0}^{\infty} x^n {(1+\frac{x^2}{\nu})^{\frac{-\nu-1}{2}}} dx$$

If n is odd, the function is an odd function so the value should be zero,

If n is even,

Let $u = 1+\frac{x^2}{\nu}$, so $x = \sqrt{(u-1) \nu}$, $dx = \frac{1}{2}\nu \frac{1}{\sqrt{(u-1) \nu}}du$

$$ \int_{1}^{\infty} ( \sqrt{(u-1) \nu})^n {u^{\frac{-\nu-1}{2}}} \nu{(\sqrt{(u-1) \nu})}^{-1}du$$

$$ \nu \int_{1}^{\infty} ( \sqrt{(u-1) \nu})^{n-1} {u^{\frac{-\nu-1}{2}}}  du$$

$$  \nu^{\frac{n+1}{2}} \int_{1}^{\infty} {(u-1)}^{\frac{n-1}{2}} {u^{\frac{-\nu-1}{2}}}  du$$

Focusing on the integral,

Let $u = w^{-1}$, then $du = -w^{-2} dw$

$$\int_{0}^{1} {(w^{-1}-1)}^{\frac{n-1}{2}} {(w^{-1})^{\frac{-\nu-1}{2}}}  w^{-2} dw$$

$$\int_{0}^{1} {(w^{-1}(1-w))}^{\frac{n-1}{2}} {w^{\frac{\nu-3}{2}}}   dw$$

$$\int_{0}^{1} {w^{\frac{1-n}{2}} (1-w)}^{\frac{n-1}{2}} {w^{\frac{\nu-3}{2}}}   dw$$

$$\int_{0}^{1} { (1-w)}^{\frac{n-1}{2}} {w^{\frac{\nu-2-n}{2}}}   dw$$

Which as it turns out to be, 
$${\bf B}(\frac{\nu-n}{2},\frac{n+1}{2})$$

We can untar it by our fantastically mathematics talent, 

$$ \frac{\Gamma(\frac{\nu+1}{2})}{\Gamma(\frac{\nu}{2})} \frac{1}{\sqrt{\nu \pi}} \nu^{\frac{n+1}{2}} {\bf B}(\frac{\nu-n}{2},\frac{n+1}{2})$$


$$ \frac{\Gamma(\frac{\nu+1}{2})}{\Gamma(\frac{\nu}{2})} \frac{1}{\sqrt{\nu \pi}} \nu^{\frac{n+1}{2}} \frac{\Gamma(\frac{\nu-n}{2})\Gamma(\frac{n+1}{2})}{\Gamma{(\frac{\nu+1}{2}})}$$

Overall,
$$  \frac{1}{\sqrt{ \pi}} \nu^{\frac{n}{2}} \frac{\Gamma(\frac{\nu-n}{2})\Gamma(\frac{n+1}{2})}{\Gamma(\frac{\nu}{2})}$$

\subsection{Examples}

1. The Student's t-distribution with  degrees of freedom can be defined as the distribution of the random variable T,
$$T = \frac{Z}{(\frac{V}{\nu})^{1/2}}$$
where

Z  is a standard normal random variable;
V  has a chi-squared distribution with $\nu$ degrees of freedom;
Z and V are independent.
Show that if T has Student (t-distribution) distribution with n degrees of freedom, then $\sigma{(T)} = \frac{n}{n-2}$ for $n>2$.

Prove:
Using the fact given above, plug in $n = 2$, 
$$  \frac{1}{\sqrt{ \pi}} \nu^{\frac{2}{2}} \frac{\Gamma(\frac{\nu-2}{2})\Gamma(\frac{3}{2})}{\Gamma(\frac{\nu}{2})}$$
By common sense, 
$$\Gamma(\frac{3}{2}) = \frac{1}{2} \Gamma(\frac{1}{2})$$

$$\Gamma(\frac{\nu}{2}) = \frac{\nu-2}{2} \Gamma(\frac{\nu-2}{2})$$

So $$  \frac{1}{\sqrt{ \pi}} \nu^{\frac{2}{2}} \frac{\Gamma(\frac{\nu-2}{2})\Gamma(\frac{3}{2})}{\Gamma(\frac{\nu}{2})} = \frac{1}{\sqrt{ \pi}} \nu\frac{\frac{1}{2}\sqrt{\pi}}{\frac{\nu-2}{2}} =\frac\nu{\nu-2}  $$

2. Recall that the skewness of X is the third moment of its standard score and its kurtosis is the fourth moment of the standard score. Find these characteristics for X having Gamma distribution, Chi-square distribution, Student distribution.

For Gamma Distribution, consider 

$$MGF(t) = \frac{1}{(1-\beta t) ^ \alpha }$$

Then 
$$MGF(t) = {(1-\beta t) ^ {-\alpha} }$$
$$MGF'(t) = \beta \alpha {(1-\beta t) ^ {-\alpha-1} }$$
$$MGF''(t) = \beta^2 \alpha ({\alpha+1}){(1-\beta t) ^ {-\alpha-2} }$$
$$MGF^{3}(t) = \beta^3 \alpha ({\alpha+1})({\alpha+2}){(1-\beta t) ^ {-\alpha-3} }$$
$$MGF^{4}(t) = \beta^4 \alpha ({\alpha+1})({\alpha+2})({\alpha+3}){(1-\beta t) ^ {-\alpha-4} }$$

$$MGF^{3}(0) = \beta^3 \alpha ({\alpha+1})({\alpha+2}) $$

$$MGF^{4}(0) =  \beta^4 \alpha ({\alpha+1})({\alpha+2})({\alpha+3}) $$

For the $\chi^2$ distribution,

$$MGF^{3}(0) = 2^3 \frac{p}{2} ({\frac{p}{2}+1})({\frac{p}{2}+2}) = p(p+2)(p+4) $$
$$MGF^{4}(0) = p(p+2)(p+4)(p+6) $$

For Student distribution, plug n = 4

$$  \frac{1}{\sqrt{ \pi}} \nu^{\frac{4}{2}} \frac{\Gamma(\frac{\nu-4}{2})\Gamma(\frac{4+1}{2})}{\Gamma(\frac{\nu}{2})}$$

$$  \frac{1}{\sqrt{ \pi}} \nu^{\frac{4}{2}} \frac{\Gamma(\frac{\nu-4}{2})\Gamma(\frac{4+1}{2})}{\Gamma(\frac{\nu}{2})}$$

More common sense, 
$$\Gamma(\frac{\nu}{2}) = \frac{\nu-2}{2} \Gamma(\frac{\nu-2}{2}) =  \frac{\nu-2}{2} \frac{\nu-4}{2} \Gamma(\frac{\nu-4}{2}) $$

$$\Gamma(\frac{4+1}{2}) = \frac{3}{2}\Gamma(\frac{2+1}{2}) = \frac{3}{2} \frac{1}{2}\Gamma(\frac{1}{2}) = \frac{3}{2} \frac{1}{2} \sqrt{\pi}$$

So, 
$$  \frac{1}{\sqrt{ \pi}} \nu^{\frac{4}{2}} \frac{\Gamma(\frac{\nu-4}{2})\Gamma(\frac{4+1}{2})}{\Gamma(\frac{\nu}{2})} =\nu^{2} \frac{ 3 }{{(\nu-2)} {(\nu-4)}} $$


3. Find the limiting distribution of the Student distribution as the number of degrees of freedom approaches $\infty$.


$$\frac{\Gamma(\frac{\nu+1}{2})}{\Gamma(\frac{\nu}{2})} \frac{1}{\sqrt{\nu \pi}} \frac{1}{(1+\frac{x^2}{\nu})^{\frac{\nu+1}{2}}}$$

As $\nu$ tends to $\infty$, we can use stirling approximation 

$$\Gamma(x) = \sqrt{\frac{2\pi}{x}}(\frac{x}{e})^{x}$$

$$\frac{\sqrt{\frac{2\pi}{\frac{\nu+1}{2}}}(\frac{\frac{\nu+1}{2}}{e})^{\frac{\nu+1}{2}}}{\sqrt{\frac{2\pi}{\frac{\nu}{2}}}(\frac{\frac{\nu}{2}}{e})^{\frac{\nu}{2}}} \frac{1}{\sqrt{\nu \pi}} {(1+\frac{x^2}{\nu})^{\frac{-\nu-1}{2}}} $$

$$=\frac{\sqrt{\frac{{\frac{\nu}{2}}}{\frac{\nu+1}{2}}}(\frac{\frac{\nu+1}{2}}{e})^{\frac{\nu+1}{2}}}{(\frac{\frac{\nu}{2}}{e})^{\frac{\nu}{2}}} \frac{1}{\sqrt{\nu \pi}} {(1+\frac{x^2}{\nu})^{\frac{-\nu-1}{2}}} $$

$$=\frac{\sqrt{\frac{{\frac{\nu}{2}}}{\frac{\nu+1}{2}}}(\frac{\frac{\nu+1}{2}}{e})^{\frac{\nu}{2}} (\frac{\frac{\nu+1}{2}}{e})^{\frac{1}{2}} }{(\frac{\frac{\nu}{2}}{e})^{\frac{\nu}{2}}} \frac{1}{\sqrt{\nu \pi}} {(1+\frac{x^2}{\nu})^{\frac{-\nu-1}{2}}} $$

$$={\sqrt{\frac{{\frac{\nu}{2}}}{\frac{\nu+1}{2}}}(\frac{\frac{\nu+1}{2}}{\frac{\nu}{2}})^{\frac{\nu}{2}} (\frac{\frac{\nu+1}{2}}{e})^{\frac{1}{2}} } \frac{1}{\sqrt{\nu \pi}} {(1+\frac{x^2}{\nu})^{\frac{-\nu-1}{2}}} $$

$$={\sqrt{\frac{{\frac{\nu}{2}}}{\frac{\nu+1}{2}}}(\frac{\frac{\nu+1}{2}}{\frac{\nu}{2}})^{\frac{\nu}{2}} (\frac{\frac{\nu+1}{2}}{\nu e})^{\frac{1}{2}} } {{ \pi}}^{-1/2} {(1+\frac{x^2}{\nu})^{\frac{-\nu-1}{2}}} $$
$$={\sqrt{\frac{{\frac{\nu}{2}}}{\frac{\nu+1}{2}}}(\frac{\frac{\nu+1}{2}}{\frac{\nu}{2}})^{\frac{\nu}{2}} (\frac{\frac{\nu+1}{2}}{\nu e})^{\frac{1}{2}} } {{ \pi}}^{-1/2} {(1+\frac{x^2}{\nu})^{\frac{-\nu}{2}}} {(1+\frac{x^2}{\nu})^{\frac{-1}{2}}} $$

As, $\nu$ tends to infinity, $\sqrt{\frac{{\frac{\nu}{2}}}{\frac{\nu+1}{2}}} = 1$, $(\frac{\frac{\nu+1}{2}}{\frac{\nu}{2}})^{\frac{\nu}{2}} = (\frac{{\nu+1}}{{\nu}})^{\frac{\nu}{2}}=(1+\frac{1}{\nu})^{\nu/2} = e^{1/2}$

$(\frac{\frac{\nu+1}{2}}{\nu e})^{\frac{1}{2}} = 2^{-1/2} e^{-1/2}$, ${(1+\frac{x^2}{\nu})^{\frac{-\nu}{2}}} = e^{-x^2/2}$,${(1+\frac{x^2}{\nu})^{\frac{-1}{2}}}  = 1$


So 

$${\sqrt{\frac{{\frac{\nu}{2}}}{\frac{\nu+1}{2}}}(\frac{\frac{\nu+1}{2}}{\frac{\nu}{2}})^{\frac{\nu}{2}} (\frac{\frac{\nu+1}{2}}{\nu e})^{\frac{1}{2}} } {{ \pi}}^{-1/2} {(1+\frac{x^2}{\nu})^{\frac{-\nu}{2}}} {(1+\frac{x^2}{\nu})^{\frac{-1}{2}}} = $$

$$1\times e^{1/2} \times2^{-1/2} e^{-1/2} \times {{ \pi}}^{-1/2} \times  e^{-x^2/2} \times 1=$$

$$\frac{e^{x^2/2}}{\sqrt{2\pi}}$$

So it approximates to the standard normal distribution as the degree of freedom goes to infinity

\end{document}
