\documentclass{article}
\usepackage[utf8]{inputenc}
\usepackage{amsmath}

\title{More Calculus Haha}
\author{waluigi120 }
\date{April 2018}

\begin{document}

\maketitle

\section{Double Exponential}

\subsection{Definition}
$$y = \frac{1}{2\sigma} e^{-\frac{|x-\mu|}{\sigma}}$$

\subsection{Moment Generating Function}
\begin{align*}\label{eq:pareto mle2}
 Mgf(t) =\int_{-\infty}^{+\infty} e^{tx} \frac{1}{2\sigma} e^{-\frac{|x-\mu|}{\sigma}} dx\\ =\int_{-\infty}^{+\infty} \frac{1}{2\sigma} e^{tx-\frac{|x-\mu|}{\sigma}}dx \\ = \int_{-\infty}^{\mu} \frac{1}{2\sigma} e^{tx-\frac{\mu - x}{\sigma}} dx \\+ \int_{\mu}^{\infty} \frac{1}{2\sigma} e^{tx-\frac{x-\mu}{\sigma}} dx \\ = \int_{-\infty}^{\mu} \frac{1}{2\sigma} e^{tx-\frac{\mu - x}{\sigma}}dx \\+ \int_{\mu}^{\infty} \frac{1}{2\sigma} e^{tx-\frac{x-\mu}{\sigma}}dx\\ = \int_{-\infty}^{\mu} \frac{1}{2\sigma} e^{\frac{\sigma tx -\mu + x}{\sigma}}dx \\+ \int_{\mu}^{\infty} \frac{1}{2\sigma} e^{\frac{\sigma tx-x+\mu}{\sigma}}dx \\ = \frac{1}{2 \sigma}( \int_{-\infty}^{\mu}  e^{\frac{\sigma tx -\mu + x}{\sigma}}dx +  \int_{\mu}^{\infty}  e^{\frac{\sigma tx-x+\mu}{\sigma}}dx )\\ = \frac{1}{ 2\sigma} ( e^{-\frac{\mu}{\sigma}} \int_{-\infty}^{\mu} e^{\frac{(\sigma t + 1) x}{\sigma}} dx +  e^{\frac{\mu}{\sigma}} \int_{\mu}^{\infty}  e^{\frac{\sigma tx-x}{\sigma}}dx )) \\ =  \frac{1}{2 \sigma} (e^{-\frac{\mu}{\sigma}} \frac{\sigma}{\sigma t + 1} e^{\frac{(\sigma t + 1) x}{\sigma}} |_{x = -\infty}^{\mu}+e^{\frac{\mu}{\sigma}} \frac{\sigma}{\sigma t - 1} e^{\frac{(\sigma t - 1) x}{\sigma}} |_{x = \mu}^{\infty} ) \\ = \frac{1}{2\sigma} (e^{-\frac{\mu}{\sigma}} \frac{\sigma}{\sigma t + 1} e^{\frac{(\sigma t + 1) \mu}{\sigma}} - e^{\frac{\mu}{\sigma}} \frac{\sigma}{\sigma t - 1} e^{\frac{(\sigma t - 1) \mu}{\sigma}} )\end{align*}
\begin{align*}
 = \frac{1}{2\sigma} ( \frac{\sigma}{\sigma t + 1} e^{\frac{(\sigma t ) \mu}{\sigma}} -  \frac{\sigma}{\sigma t - 1} e^{\frac{(\sigma t) \mu}{\sigma}} ) =   \frac{1}{1-(\sigma t)^{2} } e^{ t  \mu}
\end{align*}

\subsection{Means and Variance}

Mean is $\mu$, which is easy to show.

Variance ,easy to be proven by substitution and take derivative, is $2\sigma^2$

\section{Chi-sqruare Distribution}

\subsection{Moments}

Consider the following integration 
$$EX^n = \int_{0}^{\infty} x^n \frac{1}{\Gamma(\frac{p}2)2^{\frac{p}2}} x ^{\frac{p}2-1}e^{-\frac{x}2}dx$$

$$ = \frac{1}{\Gamma(\frac{p}2)2^{\frac{p}2}} \int_{0}^{\infty} x^n  x ^{\frac{p}2-1}e^{-\frac{x}2}dx$$

$$= \frac{1}{\Gamma(\frac{p}2)2^{\frac{p}2}} \int_{0}^{\infty}  x ^{n+\frac{p}2-1}e^{-\frac{x}2}dx$$

Let $u = \frac{x}{2}$ then $2du = dx$
$$= \frac{1}{\Gamma(\frac{p}2)2^{\frac{p}2}} \int_{0}^{\infty}  (2u)^{n+\frac{p}2-1}e^{-u}2du$$

$$= \frac{1}{\Gamma(\frac{p}2)} 2^{-\frac{p}2} 2^{n+\frac{p}2-1} \int_{0}^{\infty}  (u)^{n+\frac{p}2-1}e^{-u}2du$$
 
$$= \frac{1}{\Gamma(\frac{p}2)}  2^{n} \int_{0}^{\infty}  (u)^{n+\frac{p}2-1}e^{-u}du$$

$$EX^n= \frac{2^{n} \Gamma(n+\frac{p}2)}{\Gamma(\frac{p}2)}   $$

\subsection{Examples}

1. The Student's t-distribution with  degrees of freedom can be defined as the distribution of the random variable T,
$$T = \frac{Z}{(\frac{V}{\nu})^{1/2}}$$
where

Z  is a standard normal random variable;
V  has a chi-squared distribution with $\nu$ degrees of freedom;
Z and V are independent.
Show that if T has Student (t-distribution) distribution with n degrees of freedom, then $\sigma{(T)} = \frac{n}{n-2}$ for $n>2$.

Prove:
$$E T = E(\frac{Z}{(\frac{V}{\nu})^{1/2}})$$

Since V, Z are independent,
$$E T = {EZ} \times E(\frac{1}{(\frac{V}{\nu})^{1/2}})$$

Since $EZ = 0$, $ET = 0$

$$E T^2 = E(\frac{Z^2}{(\frac{V}{\nu})})$$

Since V, Z are independent,
$$E T^2 = ({E(Z^2)}\nu E({V}^{-1}))$$

Since $EZ^2 = 1$, $EV^{-1} = \frac{2^{-1}}{(-1+\frac{\nu}2)} = \frac{1}{\nu-2} $,

So $$E T^2 = \frac{\nu}{\nu-2}$$

2. Recall that the skewness of X is the third moment of its standard score and its kurtosis is the fourth moment of the standard score. Find these characteristics for X having Gamma distribution, Chi-square distribution, Student distribution.

For Student Distribution, Consider, 
$$E (\frac{T-\mu}{\sigma})^4 = E (\frac{T}{\sigma})^4 = (\frac{ET^4}{\sigma^4}) = E(Z^4) E(V^{-2}) \frac{\nu^{2}}{\sigma^4}$$

$$E(Z^4) E(V^{-2}) \frac{\nu^{2}}{\sigma^4} = E((\chi^2)^2) E(V^{-2}) \frac{\nu^{2}}{\sigma^4}$$

Since $$E((\chi^2)^2) = \frac{2^{2} \Gamma(2+\frac{1}2)}{\Gamma(\frac{1}2)} = 3 $$

Since $$E((V)^{-2}) = \frac{2^{-2} \Gamma(-2+\frac{\nu}2)}{\Gamma(\frac{\nu}2)} = \frac{1}{4 {(-1+\frac{\nu}2)}{(-2+\frac{\nu}2)}} =\frac{1}{(\nu-2)(\nu-4)}$$
$$\sigma^4 = (\frac{\nu}{\nu-2})^2$$

So the kurtosis for the distribution is $\frac{6}{n-4}+3$


\section{Fisher Distribution}

\subsection{Property Related to the Chi-square Distribtuion}

$F_{(a,b)} = \frac{\chi^2(a) b}{\chi^2(b) a}$

Let $x$ --$\chi^2(a)$, $y$ -- $\chi^2(b)$

The probability the variable happen is 

$$\frac{1}{\Gamma(\frac{a}2)2^{\frac{a}2}} x ^{\frac{a}2-1}e^{-\frac{x}2} \frac{1}{\Gamma(\frac{b}2)2^{\frac{b}2}} x ^{\frac{b}2-1}e^{-\frac{y}2}$$

Let $z = \frac{xb}{ya}$, then $x = \frac{a}{b} yz$, $dx = \frac{a}{b} ydz$

$$\frac{1}{\Gamma(\frac{a}2)2^{\frac{a}2}} \frac{1}{\Gamma(\frac{b}2)2^{\frac{b}2}} x ^{\frac{a}2-1} y^{\frac{b}2-1}e^{-\frac{x}2-\frac{y}{2}} $$

For the variables, consider the integration with respect to y,
$$\int_{0}^{\infty}{(\frac{a}{b} yz)^{\frac{a}2-1} y^{\frac{b}2-1}e^{-\frac{ayz}{2b}-\frac{y}{2}} \frac{a}{b}ydy}$$

$$(\frac{a}{b} z)^{\frac{a}2} z^{-1} \int_{0}^{\infty}{( y)^{\frac{a}2+\frac{b}2-1}e^{-\frac{1}{2}{(\frac{az}{b}+1)}y} dy}$$

Consider the integral, 

Let $u = \frac{1}{2}{(\frac{az}{b}+1)}y$, $y = 2(\frac{az}{b}+1)^{-1}u $

$dy = 2(\frac{az}{b}+1)^{-1}du$

$$\int_{0}^{\infty}{(2(\frac{az}{b}+1)^{-1}u)^{\frac{a}2+\frac{b}2-1}e^{-u} 2(\frac{az}{b}+1)^{-1}du}$$

$$(\frac{2}{\frac{az}{b}+1})^{\frac{a}2+\frac{b}2}\int_{0}^{\infty}{(u)^{\frac{a}2+\frac{b}2-1}e^{-u} du}$$

$$(\frac{2}{\frac{az}{b}+1})^{\frac{a}2+\frac{b}2}\Gamma(\frac{a}2+\frac{b}2)$$

Plunge back,

$$\frac{1}{\Gamma(\frac{a}2)2^{\frac{a}2}} \frac{1}{\Gamma(\frac{b}2)2^{\frac{b}2}} (\frac{a}{b} z)^{\frac{a}2} z^{-1}(\frac{2}{\frac{az}{b}+1})^{\frac{a}2+\frac{b}2}\Gamma(\frac{a}2+\frac{b}2)$$


\subsection{Moments}
nth-Moment is defined by the following integral:

$$\int_{0}^{\infty}x^n\frac{\Gamma(\frac{\nu_1+\nu_2}{2})}{\Gamma(\frac{\nu_1}{2})\Gamma(\frac{\nu_2}{2})} (\frac{\nu_1}{\nu_2})^{\frac{\nu_1}{2}} \frac{x^{\frac{\nu_1-2}{2}}}{(1+(\frac{\nu_1}{\nu_2})x)^{\frac{\nu_1+\nu_2}{2}}}dx$$

$$\frac{\Gamma(\frac{\nu_1+\nu_2}{2})}{\Gamma(\frac{\nu_1}{2})\Gamma(\frac{\nu_2}{2})} (\frac{\nu_1}{\nu_2})^{\frac{\nu_1}{2}}\int_{0}^{\infty}x^{2n/2} \frac{x^{\frac{\nu_1-2}{2}}}{(1+(\frac{\nu_1}{\nu_2})x)^{\frac{\nu_1+\nu_2}{2}}}dx$$

Focusing on the integral,

$$\int_{0}^{\infty} \frac{x^{\frac{2n+\mu_1-2}{2}}}{(1+(\frac{\nu_1}{\nu_2})x)^{\frac{\nu_1+\nu_2}{2}}}dx$$


$$\int_{0}^{\infty} \frac{x^{\frac{2n+\nu_1-2}{2}}}{(1+(\frac{\nu_1}{\nu_2})x)^{\frac{\nu_1+\nu_2}{2}}}dx$$

Let $u = 1+(\frac{\nu_1}{\nu_2})x$ then $(u-1)(\frac{\nu_2}{\nu_1})= x$

and $du(\frac{\nu_2}{\nu_1})= dx$, if $x =0,u=1$, if $x=\infty, u = \infty$

$$\int_{1}^{\infty} \frac{((u-1)(\frac{\nu_2}{\nu_1}))^{\frac{2n+\nu_1-2}{2}}}{u^{\frac{\nu_1+\nu_2}{2}}}(\frac{\nu_2}{\nu_1})du$$

$$(\frac{\nu_2}{\nu_1})^{\frac{2n+\nu_1}{2}} \int_{1}^{\infty} \frac{(u-1)^{\frac{2n+\nu_1-2}{2}}}{u^{\frac{\nu_1+\nu_2}{2}}}du$$


$$(\frac{\nu_2}{\nu_1})^{\frac{2n+\nu_1}{2}} \int_{1}^{\infty} \frac{(u-1)^{\frac{2n+\nu_1-2}{2}}}{u^{\frac{\nu_1+\nu_2}{2}}}du$$
 
Focusing on the integral, Let $v =u^{-1}$, $dv = -u^{-2} du = -v^2du$

$du = -v^{-2} dv$

$$\int_{0}^{1} \frac{(v^{-1}-1)^{\frac{2n+\nu_1-2}{2}}}{{v^{-1}}^{\frac{\nu_1+\nu_2}{2}}}v^{-2} dv$$

$$\int_{0}^{1} (\frac{1-v}{v})^{\frac{2n+\nu_1-2}{2}}v^{-2} {{{v}^{\frac{\nu_1+\nu_2}{2}}}} dv$$

$$\int_{0}^{1} ({1-v})^{\frac{2n+\nu_1-2}{2}}v^{\frac{-2n-\nu_1+2}{2}}v^{-2} {{{v}^{\frac{\nu_1+\nu_2}{2}}}} dv$$

$$\int_{0}^{1} ({1-v})^{\frac{2n+\nu_1-2}{2}}v^{\frac{\nu_2-2n-2}{2}}{} dv$$

Which as it turns out to be, 
$${\bf B}(\frac{\nu_2-2n}{2},\frac{2n+\nu_1}{2})$$

Plug in back,

$$\frac{\Gamma(\frac{\nu_1+\nu_2}{2})}{\Gamma(\frac{\nu_1}{2})\Gamma(\frac{\nu_2}{2})} (\frac{\nu_1}{\nu_2})^{\frac{\nu_1}{2}}(\frac{\nu_2}{\nu_1})^{\frac{2n+\nu_1}{2}} {\bf B}(\frac{\nu_2-2n}{2},\frac{2n+\nu_1}{2})$$

Which will be,

$$\frac{\Gamma(\frac{\nu_2-2n}{2})\Gamma(\frac{2n+\nu_1}{2})}{\Gamma(\frac{\nu_1}{2})\Gamma(\frac{\nu_2}{2})} (\frac{\nu_2}{\nu_1})^{n} $$

\subsection{Examples}


\section{T-distribution}

\subsection{Moments}
The moments by T-distribution is defined by,

$$\int_{-\infty}^{\infty}\frac{\Gamma(\frac{\nu+1}{2})}{\Gamma(\frac{\nu}{2})} \frac{1}{\sqrt{\nu \pi}} \frac{x^n}{(1+\frac{x^2}{\nu})^{\frac{\nu+1}{2}}}dx$$

$$\frac{\Gamma(\frac{\nu+1}{2})}{\Gamma(\frac{\nu}{2})} \frac{1}{\sqrt{\nu \pi}} \int_{-\infty}^{\infty} \frac{x^n}{(1+\frac{x^2}{\nu})^{\frac{\nu+1}{2}}}dx$$

Focusing on the integral, 

$$\int_{-\infty}^{\infty} \frac{x^n}{(1+\frac{x^2}{\nu})^{\frac{\nu+1}{2}}}dx$$

Can be rewritten as,

$$2 \int_{0}^{\infty} x^n {(1+\frac{x^2}{\nu})^{\frac{-\nu-1}{2}}} dx$$

If n is odd, the function is an odd function so the value should be zero,

If n is even,

Let $u = 1+\frac{x^2}{\nu}$, so $x = \sqrt{(u-1) \nu}$, $dx = \frac{1}{2}\nu \frac{1}{\sqrt{(u-1) \nu}}du$

$$ \int_{1}^{\infty} ( \sqrt{(u-1) \nu})^n {u^{\frac{-\nu-1}{2}}} \nu{(\sqrt{(u-1) \nu})}^{-1}du$$

$$ \nu \int_{1}^{\infty} ( \sqrt{(u-1) \nu})^{n-1} {u^{\frac{-\nu-1}{2}}}  du$$

$$  \nu^{\frac{n+1}{2}} \int_{1}^{\infty} {(u-1)}^{\frac{n-1}{2}} {u^{\frac{-\nu-1}{2}}}  du$$

Focusing on the integral,

Let $u = w^{-1}$, then $du = -w^{-2} dw$

$$\int_{0}^{1} {(w^{-1}-1)}^{\frac{n-1}{2}} {(w^{-1})^{\frac{-\nu-1}{2}}}  w^{-2} dw$$

$$\int_{0}^{1} {(w^{-1}(1-w))}^{\frac{n-1}{2}} {w^{\frac{\nu-3}{2}}}   dw$$

$$\int_{0}^{1} {w^{\frac{1-n}{2}} (1-w)}^{\frac{n-1}{2}} {w^{\frac{\nu-3}{2}}}   dw$$

$$\int_{0}^{1} { (1-w)}^{\frac{n-1}{2}} {w^{\frac{\nu-2-n}{2}}}   dw$$

Which as it turns out to be, 
$${\bf B}(\frac{\nu-n}{2},\frac{n+1}{2})$$

We can untar it by our fantastically mathematics talent, 

$$ \frac{\Gamma(\frac{\nu+1}{2})}{\Gamma(\frac{\nu}{2})} \frac{1}{\sqrt{\nu \pi}} \nu^{\frac{n+1}{2}} {\bf B}(\frac{\nu-n}{2},\frac{n+1}{2})$$


$$ \frac{\Gamma(\frac{\nu+1}{2})}{\Gamma(\frac{\nu}{2})} \frac{1}{\sqrt{\nu \pi}} \nu^{\frac{n+1}{2}} \frac{\Gamma(\frac{\nu-n}{2})\Gamma(\frac{n+1}{2})}{\Gamma{(\frac{\nu+1}{2}})}$$

Overall,
$$  \frac{1}{\sqrt{ \pi}} \nu^{\frac{n}{2}} \frac{\Gamma(\frac{\nu-n}{2})\Gamma(\frac{n+1}{2})}{\Gamma(\frac{\nu}{2})}$$

\subsection{Examples}

1. The Student's t-distribution with  degrees of freedom can be defined as the distribution of the random variable T,
$$T = \frac{Z}{(\frac{V}{\nu})^{1/2}}$$
where

Z  is a standard normal random variable;
V  has a chi-squared distribution with $\nu$ degrees of freedom;
Z and V are independent.
Show that if T has Student (t-distribution) distribution with n degrees of freedom, then $\sigma{(T)} = \frac{n}{n-2}$ for $n>2$.

Prove:
Using the fact given above, plug in $n = 2$, 
$$  \frac{1}{\sqrt{ \pi}} \nu^{\frac{2}{2}} \frac{\Gamma(\frac{\nu-2}{2})\Gamma(\frac{3}{2})}{\Gamma(\frac{\nu}{2})}$$
By common sense, 
$$\Gamma(\frac{3}{2}) = \frac{1}{2} \Gamma(\frac{1}{2})$$

$$\Gamma(\frac{\nu}{2}) = \frac{\nu-2}{2} \Gamma(\frac{\nu-2}{2})$$

So $$  \frac{1}{\sqrt{ \pi}} \nu^{\frac{2}{2}} \frac{\Gamma(\frac{\nu-2}{2})\Gamma(\frac{3}{2})}{\Gamma(\frac{\nu}{2})} = \frac{1}{\sqrt{ \pi}} \nu\frac{\frac{1}{2}\sqrt{\pi}}{\frac{\nu-2}{2}} =\frac\nu{\nu-2}  $$

2. Recall that the skewness of X is the third moment of its standard score and its kurtosis is the fourth moment of the standard score. Find these characteristics for X having Gamma distribution, Chi-square distribution, Student distribution.

For Gamma Distribution, consider 

$$MGF(t) = \frac{1}{(1-\beta t) ^ \alpha }$$

Then 
$$MGF(t) = {(1-\beta t) ^ {-\alpha} }$$
$$MGF'(t) = \beta \alpha {(1-\beta t) ^ {-\alpha-1} }$$
$$MGF''(t) = \beta^2 \alpha ({\alpha+1}){(1-\beta t) ^ {-\alpha-2} }$$
$$MGF^{3}(t) = \beta^3 \alpha ({\alpha+1})({\alpha+2}){(1-\beta t) ^ {-\alpha-3} }$$
$$MGF^{4}(t) = \beta^4 \alpha ({\alpha+1})({\alpha+2})({\alpha+3}){(1-\beta t) ^ {-\alpha-4} }$$

$$MGF^{3}(0) = \beta^3 \alpha ({\alpha+1})({\alpha+2}) $$

$$MGF^{4}(0) =  \beta^4 \alpha ({\alpha+1})({\alpha+2})({\alpha+3}) $$

For the $\chi^2$ distribution,

$$MGF^{3}(0) = 2^3 \frac{p}{2} ({\frac{p}{2}+1})({\frac{p}{2}+2}) = p(p+2)(p+4) $$
$$MGF^{4}(0) = p(p+2)(p+4)(p+6) $$

For Student distribution, plug n = 4

$$  \frac{1}{\sqrt{ \pi}} \nu^{\frac{4}{2}} \frac{\Gamma(\frac{\nu-4}{2})\Gamma(\frac{4+1}{2})}{\Gamma(\frac{\nu}{2})}$$

$$  \frac{1}{\sqrt{ \pi}} \nu^{\frac{4}{2}} \frac{\Gamma(\frac{\nu-4}{2})\Gamma(\frac{4+1}{2})}{\Gamma(\frac{\nu}{2})}$$

More common sense, 
$$\Gamma(\frac{\nu}{2}) = \frac{\nu-2}{2} \Gamma(\frac{\nu-2}{2}) =  \frac{\nu-2}{2} \frac{\nu-4}{2} \Gamma(\frac{\nu-4}{2}) $$

$$\Gamma(\frac{4+1}{2}) = \frac{3}{2}\Gamma(\frac{2+1}{2}) = \frac{3}{2} \frac{1}{2}\Gamma(\frac{1}{2}) = \frac{3}{2} \frac{1}{2} \sqrt{\pi}$$

So, 
$$  \frac{1}{\sqrt{ \pi}} \nu^{\frac{4}{2}} \frac{\Gamma(\frac{\nu-4}{2})\Gamma(\frac{4+1}{2})}{\Gamma(\frac{\nu}{2})} =\nu^{2} \frac{ 3 }{{(\nu-2)} {(\nu-4)}} $$


3. Find the limiting distribution of the Student distribution as the number of degrees of freedom approaches $\infty$.


$$\frac{\Gamma(\frac{\nu+1}{2})}{\Gamma(\frac{\nu}{2})} \frac{1}{\sqrt{\nu \pi}} \frac{1}{(1+\frac{x^2}{\nu})^{\frac{\nu+1}{2}}}$$

As $\nu$ tends to $\infty$, we can use stirling approximation 

$$\Gamma(x) = \sqrt{\frac{2\pi}{x}}(\frac{x}{e})^{x}$$

$$\frac{\sqrt{\frac{2\pi}{\frac{\nu+1}{2}}}(\frac{\frac{\nu+1}{2}}{e})^{\frac{\nu+1}{2}}}{\sqrt{\frac{2\pi}{\frac{\nu}{2}}}(\frac{\frac{\nu}{2}}{e})^{\frac{\nu}{2}}} \frac{1}{\sqrt{\nu \pi}} {(1+\frac{x^2}{\nu})^{\frac{-\nu-1}{2}}} $$

$$=\frac{\sqrt{\frac{{\frac{\nu}{2}}}{\frac{\nu+1}{2}}}(\frac{\frac{\nu+1}{2}}{e})^{\frac{\nu+1}{2}}}{(\frac{\frac{\nu}{2}}{e})^{\frac{\nu}{2}}} \frac{1}{\sqrt{\nu \pi}} {(1+\frac{x^2}{\nu})^{\frac{-\nu-1}{2}}} $$

$$=\frac{\sqrt{\frac{{\frac{\nu}{2}}}{\frac{\nu+1}{2}}}(\frac{\frac{\nu+1}{2}}{e})^{\frac{\nu}{2}} (\frac{\frac{\nu+1}{2}}{e})^{\frac{1}{2}} }{(\frac{\frac{\nu}{2}}{e})^{\frac{\nu}{2}}} \frac{1}{\sqrt{\nu \pi}} {(1+\frac{x^2}{\nu})^{\frac{-\nu-1}{2}}} $$

$$={\sqrt{\frac{{\frac{\nu}{2}}}{\frac{\nu+1}{2}}}(\frac{\frac{\nu+1}{2}}{\frac{\nu}{2}})^{\frac{\nu}{2}} (\frac{\frac{\nu+1}{2}}{e})^{\frac{1}{2}} } \frac{1}{\sqrt{\nu \pi}} {(1+\frac{x^2}{\nu})^{\frac{-\nu-1}{2}}} $$

$$={\sqrt{\frac{{\frac{\nu}{2}}}{\frac{\nu+1}{2}}}(\frac{\frac{\nu+1}{2}}{\frac{\nu}{2}})^{\frac{\nu}{2}} (\frac{\frac{\nu+1}{2}}{\nu e})^{\frac{1}{2}} } {{ \pi}}^{-1/2} {(1+\frac{x^2}{\nu})^{\frac{-\nu-1}{2}}} $$
$$={\sqrt{\frac{{\frac{\nu}{2}}}{\frac{\nu+1}{2}}}(\frac{\frac{\nu+1}{2}}{\frac{\nu}{2}})^{\frac{\nu}{2}} (\frac{\frac{\nu+1}{2}}{\nu e})^{\frac{1}{2}} } {{ \pi}}^{-1/2} {(1+\frac{x^2}{\nu})^{\frac{-\nu}{2}}} {(1+\frac{x^2}{\nu})^{\frac{-1}{2}}} $$

As, $\nu$ tends to infinity, $\sqrt{\frac{{\frac{\nu}{2}}}{\frac{\nu+1}{2}}} = 1$, $(\frac{\frac{\nu+1}{2}}{\frac{\nu}{2}})^{\frac{\nu}{2}} = (\frac{{\nu+1}}{{\nu}})^{\frac{\nu}{2}}=(1+\frac{1}{\nu})^{\nu/2} = e^{1/2}$

$(\frac{\frac{\nu+1}{2}}{\nu e})^{\frac{1}{2}} = 2^{-1/2} e^{-1/2}$, ${(1+\frac{x^2}{\nu})^{\frac{-\nu}{2}}} = e^{-x^2/2}$,${(1+\frac{x^2}{\nu})^{\frac{-1}{2}}}  = 1$


So 

$${\sqrt{\frac{{\frac{\nu}{2}}}{\frac{\nu+1}{2}}}(\frac{\frac{\nu+1}{2}}{\frac{\nu}{2}})^{\frac{\nu}{2}} (\frac{\frac{\nu+1}{2}}{\nu e})^{\frac{1}{2}} } {{ \pi}}^{-1/2} {(1+\frac{x^2}{\nu})^{\frac{-\nu}{2}}} {(1+\frac{x^2}{\nu})^{\frac{-1}{2}}} = $$

$$1\times e^{1/2} \times2^{-1/2} e^{-1/2} \times {{ \pi}}^{-1/2} \times  e^{-x^2/2} \times 1=$$

$$\frac{e^{x^2/2}}{\sqrt{2\pi}}$$

So it approximates to the standard normal distribution as the degree of freedom goes to infinity

\section{Random Distribtuion}

\subsection{Examples}

1. Suppose that X and Y are independent random variables, each with the exponential distribution with rate parameter r. Find the distribution of X/Y.

The probability is 
$$\frac{1}{r} e^{-\frac{x}{r}} \frac{1}{r} e^{-\frac{y}{r}} $$

$$\frac{1}{r^2} e^{-\frac{(x+y)}{r}}   $$

Let $z = \frac{x}{y}$,  $dz = \frac{dx}{y}$,$ydz = dx$, $\frac{dx}{dz} = y$

$$\frac{1}{r^2} e^{-\frac{(x+y)}{r}}  dx =  \frac{1}{r^2} e^{-\frac{(z+1)y}{r}}  ydz $$

$$\int_{0}^{\infty}\frac{1}{r^2} e^{-\frac{(z+1)y}{r}}  y dy $$ 
Let $-u = -\frac{(z+1)y}{r}$, $-du = -\frac{(z+1)dy}{r}$, $dy = \frac{rdu}{(z+1)}$, $y = \frac{ru}{(z+1)} $

So 

$$\int_{0}^{\infty}\frac{1}{r^2} e^{-u}  \frac{ru}{(z+1)} \frac{rdu}{(z+1)}$$ 

$$\int_{0}^{\infty} e^{-u}  \frac{u}{(z+1)^2} du $$

$$pdf(z)=\frac{1}{(z+1)^2}$$


\end{document}
